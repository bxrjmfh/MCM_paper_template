\documentclass[13pt]{ctexart}
\usepackage{geometry}
% memo 的设置页眉页脚的包
\usepackage{graphicx}
\usepackage{color}
\usepackage{lipsum}
%颜色color使用该宏包
\usepackage{changepage}
%有adjustwidth被使用
\usepackage{fancyhdr}
\pagestyle{fancy}
%设置页眉页脚的
\usepackage{lastpage}
%引入页数

% 设置列表缩进
\usepackage[shortlabels]{enumitem}

% 设置修改默认的section标题大小
\usepackage{titlesec}
\titleformat*{\section}{\LARGE}
\titleformat*{\subsection}{\Large}
\titleformat*{\subsubsection}{\Large}

\usepackage{amsmath}
% 设置表格的列格式否则报错
\usepackage{array}
% 三线表宏包
\usepackage{booktabs}
%\toprule、\midrule、\bottomrule等表格命令
\renewcommand{\figurename}{Figure}
\renewcommand{\tablename}{Table}
\renewcommand{\contentsname}{Content}
% 设置列表缩进
\usepackage{etoolbox}
\usepackage{newtxtext}
%引入newtxtext之后reference之前的中文字样”参考文献“消失了
\patchcmd{\thebibliography}{\section*{\refname}}{}{}{}
% 引入网站作为参考文献,处理引用网页时过长不换行的问题
\usepackage{url}
\usepackage{caption}
% 生成大图的标题包
\usepackage{subfigure}
% 生成子图的包

\newcommand\s[1]{\textbf{\section{#1}}}
\newcommand\subs[1]{\textbf{\subsection{#1}}}
\newcommand\subss[1]{\textbf{\subsubsection{#1}}}
\newcommand\pic{\centerline{\textcolor{red}{picture!}}}
\begin{document}
    \newgeometry{left=1in,right=0.75in,top=1in,bottom=1in}
    % 设置页面的边距,有多种设置方式
    % 设置文档字体,指定粗体和正常字体
    % \setmainfont{Times New Roman}[
        % UprightFont  =Times New Roman,
        % BoldFont  =Times New Roman Negreta,]
    \thispagestyle{empty}

    \vspace*{-16ex}
    % 设置垂直间距(往上偏移的位置)
    \centerline{\begin{tabular}{ccc}%这里的ccc代表三个box都是居中的形式来写的
                                    %也可以搞成……tabular}{3*{c}}                            
        \parbox[t]{0.3\linewidth}{\begin{center}\textbf{Problem Chosen}\\ \Large\textcolor{red}{ABCDEF}\end{center}}
        %red是在color宏包中定义的,故需要使用相应的pacage
        & \parbox[t]{0.3\linewidth}{\begin{center}\textbf{2021\\ MCM/ICM\\ Summary Sheet}\end{center}}
        & \parbox[t]{0.3\linewidth}{\begin{center}\textbf{Team Control Number}\\ \Large \textcolor{red}{111111}\end{center}}	\\
            \hline%相当于是表格的横线
    \end{tabular}}%tabular制表的代码,

    \vspace*{7ex}%标题放这里
    {\centering\fontsize{18}{16}\selectfont\textbf{{This Is the Title}}

    \vspace{20pt}

    \fontsize{13}{10}\selectfont\textbf{{Summary}}\par}%par为分段或者空出一行
    \vspace{15pt}

    \fontsize{13}{12.5}\selectfont

\begin{adjustwidth}{1cm}{1cm}%使用这玩意儿可以轻松控制页边距
    \indent { }{ }{ }{ }{ }{ }
    
\vspace{15pt}
\textbf{key words}: key 1 ; key 2 ; key 3
\end{adjustwidth}
    

% 目录页
\newpage
\thispagestyle{empty}
\tableofcontents


\newpage
% 目录页后面是第一页
\setcounter{page}{1}
%将page这一计数器设置为1,亦即从第一页开始计数
%还可以从其他的操作来增加或者减少



% 开始写正文
% 设置正文的页边距
\newgeometry{top=3cm, left=3.5cm, right=3.5cm}
% 设置正文的页眉页脚
\fancyhf{}
\fancyhead[C]{ }
% 此处修改右上角页码
\fancyhead[R]{Page \thepage\ of \pageref{LastPage} }
\fancyhead[L]{Team \# 1917694}%编辑队号
\fancyfoot[C]{\bfseries\thepage}


\textbf{\section{Introduction}}
\textbf{\subsection{Restatement of the Problem}}

\textbf{\subsection{Our Works}}

\textbf{\section{Assumptions and Notations}}

\vspace*{10pt}
\textbf{\subsection{Assumptions}}

\textbf{\subsection{Notations}}
Here are all the notations and their meanings in this paper.

\newpage

\textbf{\section{Model Construction}}

\textbf{\subsection{xxxx Model}}

\textbf{\subsection{xxxx Model}}

\textbf{\subsection{xxxx Model}}

\subsection{Modification to xxx Model}

\subsubsection{xx way xx way}

\subsubsection{Modifying Our Model with xxx way}

\subsubsection{Analysis of the Modified Model}

\section{Model Simulation and Analysis}%模型的求解和分析

\subsection{Describing Characteristics of XXX Model}

\subsection{XXX}

\section{Strengths and Weakness}

\subsection{Strengths}

\subsection{Weakness}

\section{Sensitivity Analysis}

\section{Conclusion}

\newpage

\textbf{\section*{References}\addcontentsline{toc}{section}{References}}
\fancyhf{}
\fancyhead[R]{ }
\fancyhead[L]{ }
\tolerance=5
\bibliography{books}

\bibliographystyle{IEEEtran}

\end{document}