\documentclass[13pt]{ctexart}
\usepackage{geometry}
\usepackage{graphicx}
\usepackage{color}

\renewcommand{\figurename}{Figure}
\renewcommand{\tablename}{Table}
\renewcommand{\contentsname}{Content}

\begin{document}
    \newgeometry{left=1in,right=0.75in,top=1in,bottom=1in}
    % 设置页面的边距,有多种设置方式
    \setmainfont{Times New Roman}
    \thispagestyle{empty}

    \vspace*{-16ex}
    % 设置垂直间距(往上偏移的位置)
    \centerline{\begin{tabular}{ccc}%这里的ccc代表三个box都是居中的形式来写的
                                    %也可以搞成……tabular}{3*{c}}                            
        \parbox[t]{0.3\linewidth}{\begin{center}\textbf{Problem Chosen}\\ \Large\textcolor{red}{ABCDEF}\end{center}}
        %red是在color宏包中定义的,故需要使用相应的pacage
        & \parbox[t]{0.3\linewidth}{\begin{center}\textbf{2021\\ MCM/ICM\\ Summary Sheet}\end{center}}
        & \parbox[t]{0.3\linewidth}{\begin{center}\textbf{Team Control Number}\\ \Large \textcolor{red}{111111}\end{center}}	\\
            \hline%相当于是表格的横线
    \end{tabular}}%tabular制表的代码,

    \vspace*{3ex}%标题放这里
    {\centering\fontsize{18}{16}\selectfont\textbf{{This Is the Title}}
    \vspace{10pt}

\end{document}