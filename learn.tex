\documentclass[13pt]{ctexart}
\usepackage{geometry}
\usepackage{graphicx}
\usepackage{color}
\usepackage{lipsum}
%颜色color使用该宏包
\usepackage{changepage}
%有adjustwidth被使用
\usepackage{fancyhdr}
\pagestyle{fancy}
%设置页眉页脚的
\usepackage{lastpage}
%引入页数

% 设置列表缩进
\usepackage[shortlabels]{enumitem}

% 设置修改默认的section标题大小
\usepackage{titlesec}
\titleformat*{\section}{\LARGE}
\titleformat*{\subsection}{\Large}
\titleformat*{\subsubsection}{\Large}

\usepackage{amsmath}
% 设置表格的列格式否则报错
\usepackage{array}
% 三线表宏包
\usepackage{booktabs}
%\toprule、\midrule、\bottomrule等表格命令
\renewcommand{\figurename}{Figure}
\renewcommand{\tablename}{Table}
\renewcommand{\contentsname}{Content}

\begin{document}
    \newgeometry{left=1in,right=0.75in,top=1in,bottom=1in}
    % 设置页面的边距,有多种设置方式
    \setmainfont{Times New Roman}
    \thispagestyle{empty}

    \vspace*{-16ex}
    % 设置垂直间距(往上偏移的位置)
    \centerline{\begin{tabular}{ccc}%这里的ccc代表三个box都是居中的形式来写的
                                    %也可以搞成……tabular}{3*{c}}                            
        \parbox[t]{0.3\linewidth}{\begin{center}\textbf{Problem Chosen}\\ \Large\textcolor{red}{ABCDEF}\end{center}}
        %red是在color宏包中定义的,故需要使用相应的pacage
        & \parbox[t]{0.3\linewidth}{\begin{center}\textbf{2021\\ MCM/ICM\\ Summary Sheet}\end{center}}
        & \parbox[t]{0.3\linewidth}{\begin{center}\textbf{Team Control Number}\\ \Large \textcolor{red}{111111}\end{center}}	\\
            \hline%相当于是表格的横线
    \end{tabular}}%tabular制表的代码,

    \vspace*{7ex}%标题放这里
    {\centering\fontsize{18}{16}\selectfont\textbf{{This Is the Title}}

    \vspace{20pt}

    \fontsize{13}{10}\selectfont\textbf{{Summary}}\par}%par为分段或者空出一行
    \vspace{15pt}

    \fontsize{13}{12.5}\selectfont

\begin{adjustwidth}{1cm}{1cm}%使用这玩意儿可以轻松控制页边距
    \indent { }{ }{ }{ }{ }{ }
    \lipsum[40]

    \lipsum[40]

    \lipsum[40]


\vspace{15pt}
\textbf{key words}: key 1 ; key 2 ; key 3
\end{adjustwidth}
    
% 开始写 memo 信
% 更换字体为 palatino 也可以不换
\setmainfont{texgyrepagella-regular.otf} 
\newpage
\newgeometry{left=3.5cm,right=3.5cm}
\thispagestyle{empty}

{\centering\fontsize{18pt}{14pt}\selectfont \textbf{MEMO}\par}

\vspace{8pt}

\noindent FROM: Team {} 114514,MCM C

\noindent To: The group of Governors

\noindent Date: January 28, 2019%记得改日期等
\vspace{15pt}

Dear Officials:

\lipsum[40]

%\thispagestyle{empty}
% 若信多出一页,则清理页眉页脚
%使用thispagestyle指定页码的格式,设置为empty之后就可以实现把该页的页码清除
% 信的结尾
\vspace{15pt}

{\raggedleft%右对齐
Sincerely yours

MCM C Team 1917694\par
}

% 目录页
\newpage
\thispagestyle{empty}
\tableofcontents


\newpage
% 目录页后面是第一页
\setcounter{page}{1}
%将page这一计数器设置为1,亦即从第一页开始计数
%还可以从其他的操作来增加或者减少



% 开始写正文
% 设置正文的页边距
\newgeometry{top=3cm, left=3.5cm, right=3.5cm}
% 设置正文的页眉页脚
\fancyhf{}
\fancyhead[C]{ }
% 此处修改右上角页码
\fancyhead[R]{Page \thepage\ of \pageref{LastPage} }
\fancyhead[L]{Team \# 1917694}%编辑队号
\fancyfoot[C]{\bfseries\thepage}

\textbf{\section{Introduction}}
\textbf{\subsection{Restatement of the Problem}}


\begin{itemize}
    \item \lipsum[4]
\end{itemize}

\textbf{\subsection{Our Works}}

\begin{itemize}
    \item \lipsum[4]
\end{itemize}

\textbf{\section{Assumptions and Notations}}
\lipsum[40]
\vspace*{10pt}
\textbf{\subsection{Assumptions}}

\begin{enumerate}
    \item \lipsum[2]
\end{enumerate}%有序列表

\begin{itemize}
    \item \lipsum[4]
\end{itemize}%无序列表


\textbf{\subsection{Notations}}
Here are all the notations and their meanings in this paper.
\begin{table}[h]
	\centering
	\vspace{3pt}
	\begin{tabular}{>{\centering\arraybackslash}p{5em}>{\centering\arraybackslash}p{30em}}
	\toprule % 绘制第一条线
    %\specialrule{<rule width>}{<above space>}{<below space>}
    %eg:\toprule    == \specialrule{.08em}{0pt} {.65ex}
    %可以更改参数来创建不同粗细的三线表
    Symbol & Meaning \\ \midrule
	$t$ & Time \\
	$N$ & Total reported opioid cases\\
	$N_t$ & Total reported drug cases\\
	$\lambda$ & Average cases induced by a single case\\
	$A_t$ & Status at $t$ \\ 
	$E$ & Set containing socio-economic factors with high correlation $t$ \\
	$T$ & Transition matrix\\ 
	$i(t)$ & Proportion of opioid cases in all drug cases at $t$ \\
	$\mu_1$ & Average number of drug cases induced by an existing drug case \\
	$\mu_2$ & Number of opioid cases induced among all drug cases \\
	$\gamma$ & Drug spread slow down factor \\
	$i_0$ & Status at $t$ \\ 
	$H$ & Information Entropy \\ 
	$p_0$ & Initial number of drug cases\\
	
	\bottomrule
	\end{tabular}
\end{table}
\newpage
\section{Using of the Table}
\vspace{20pt}
\centerline{\textcolor{red}{\Huge\textbf{goto learn tabular}}}
\vspace{10pt}
%goto learn tabular

\textbf{\section{Model Construction}}
\lipsum[4]
\begin{equation}
	T \cdot A_{t-1}=A_t \implies T=A_t \cdot A_{t-1}^{+}
	\label{eq_pagerank}
\end{equation}

\textbf{\subsection{xxxx Model}}
\lipsum[4]
\begin{equation}
	T \cdot A_{t-1}=A_t \implies T=A_t \cdot A_{t-1}^{+}
	\label{eq_pagerank}
\end{equation}

\textbf{\subsection{xxxx Model}}
\lipsum[4]
\begin{equation}
	T \cdot A_{t-1}=A_t \implies T=A_t \cdot A_{t-1}^{+}
	\label{eq_pagerank}
\end{equation}

\textbf{\subsection{xxxx Model}}
\lipsum[4]
\begin{equation}
	T \cdot A_{t-1}=A_t \implies T=A_t \cdot A_{t-1}^{+}
	\label{eq_pagerank}
\end{equation}
\vspace{10pt}
\centerline{\textcolor{red}{goto learn equation}}
\vspace{10pt}






\end{document}