\documentclass{article}
\usepackage{ctex}
\usepackage{amsmath}
% 设置表格的列格式否则报错
\usepackage{array}
% 三线表宏包
\usepackage{booktabs}
\usepackage{graphicx}
%\toprule、\midrule、\bottomrule等表格命令
\newcommand\s[1]{\section{#1}}
\newcommand\subs[1]{\subsection{#1}}
\newcommand\subsubs[1]{\subsubsection{#1}}

\
\begin{document}
	
\begin{table*}\centering
%\arraystretch{1.3}
\setlength{\leftskip}{0pt plus 1fil minus \marginparwidth}%通过这三行来改变三线表的对齐方式和宽度
\setlength{\rightskip}{\leftskip}
\resizebox{1.2\textwidth}{!}{
\begin{tabular}{@{}rrrrcrrrcrrr@{}}\toprule
& \multicolumn{3}{c}{$ w = 8$ } & \phantom{abc}& \multicolumn{3}{c}{$ w = 16$ } &
\phantom{abc} & \multicolumn{3}{c}{$ w = 32$ }\\
\cmidrule{2-4} \cmidrule{6-8} \cmidrule{10-12}
& $ t=0$  & $ t=1$  & $ t=2$  && $ t=0$  & $ t=1$  & $ t=2$  && $ t=0$  & $ t=1$  & $ t=2$ \\ \midrule
$ dir=1$ \\
$ c$  & 0.0790 & 0.1692 & 0.2945 && 0.3670 & 0.7187 & 3.1815 && -1.0032 & -1.7104 & -21.7969\\
$ c$  & -0.8651& 50.0476& 5.9384&& -9.0714& 297.0923& 46.2143&& 4.3590& 34.5809& 76.9167\\
$ c$  & 124.2756& -50.9612& -14.2721&& 128.2265& -630.5455& -381.0930&& -121.0518& -137.1210& -220.2500\\
$ dir=0$ \\
$ c$  & 0.0357& 1.2473& 0.2119&& 0.3593& -0.2755& 2.1764&& -1.2998& -3.8202& -1.2784\\
$ c$  & -17.9048& -37.1111& 8.8591&& -30.7381& -9.5952& -3.0000&& -11.1631& -5.7108& -15.6728\\
$ c$  & 105.5518& 232.1160& -94.7351&& 100.2497& 141.2778& -259.7326&& 52.5745& 10.1098& -140.2130\\
\bottomrule
\end{tabular}}
\caption{Caption}
\end{table*}

\begin{table}[t]
\centering
\begin{tabular}{ccc}
\toprule
\multicolumn{1}{c}{\textit{Author}}
  &&\multicolumn{1}{c}{\textit{Piece}}  \\ \midrule
Bach            &&Cello Suite \\
\bottomrule
  \end{tabular}
\caption{caption}


  \end{table}

  \begin{table}[h]
  \centering
  \vspace{3pt}
  \centering
\setlength{\leftskip}{0pt plus 1fil minus \marginparwidth}%通过这三行来改变三线表的对齐方式和宽度
\setlength{\rightskip}{\leftskip}
\resizebox{1.2\textwidth}{!}{
  \begin{tabular}{>{ \centering \arraybackslash}p{5em}>{ \centering \arraybackslash}p{30em}}
  \toprule % 绘制第一条线
	 A & b    \\
  \midrule
   t & DW  \\ 
  \bottomrule
  \end{tabular}}
  \caption{caption}
  \end{table}

  \begin{table}[h]
  \centering
  \vspace{3pt}
  \centering
	\setlength{\leftskip}{0pt plus 1fil minus \marginparwidth}%通过这三行来改变三线表的对齐方式和宽度
	\setlength{\rightskip}{\leftskip}
	\resizebox{1.2\textwidth}{!}{
  \begin{tabular}{>{ \centering \arraybackslash}p{5em}>{ \centering \arraybackslash}p{30em}}
  \toprule % 绘制第一条线
	 A & b\\
  \midrule
   t & DW  \\ 
  \bottomrule
  \end{tabular}}
  \caption{{caption:default}}
  \end{table}
    
\end{document}